
% Pacotes básicos 
\usepackage{lmodern}

\usepackage[utf8]{inputenc}
\usepackage[brazil]{babel}
\usepackage[T1]{fontenc}

\usepackage{indentfirst}
\usepackage{microtype}

\usepackage{color}
\usepackage{graphicx}

\usepackage{pdfpages}

% Pacotes matemáticos 
\usepackage{amsmath}
\usepackage{amsfonts}
\usepackage{amssymb}
\usepackage{amsthm}
\usepackage{listings,dependece/listings-rust}
% Pacotes de citações
\usepackage[brazilian,hyperpageref]{backref}

% ---------------------------------------------------------
% Selecione um estilo de referência

% Estilo de referência da ABNT
\usepackage[alf]{abntex2cite}

% Outros estilos de referências
%\bibliographystyle{abbrv}
% ---------------------------------------------------------

% Cada elemento tem sua própria numeração
%\newtheorem{teorema}{Teorema}[section]
%\newtheorem{corolario}{Corolário}[section]
%\newtheorem{definicao}{Definição}[section]
%\newtheorem{exemplo}{Exemplo}[section]

% Os elementos compartilham a mesma numeração


% CONFIGURAÇÕES DE PACOTES
\renewcommand{\backrefpagesname}{Citado na(s) página(s):~}
\renewcommand{\backref}{}
\renewcommand*{\backrefalt}[4]{
	\ifcase #1 %
		Nenhuma citação no texto.%
	\or
		Citado na página #2.%
	\else
		Citado #1 vezes nas páginas #2.%
	\fi}%

% Informações do PDF
\makeatletter
\hypersetup{
     	%pagebackref=true,
		pdftitle={\@title}, 
		pdfauthor={\@author},
    	pdfsubject={\imprimirpreambulo},
	    pdfcreator={LaTeX with abnTeX2},
		pdfkeywords={abnt}{latex}{abntex}{abntex2}{trabalho acadêmico},
		colorlinks=true,
    	linkcolor=blue,
    	citecolor=blue,
    	filecolor=magenta,
		urlcolor=blue,
		bookmarksdepth=4
}
\makeatother
\usepackage{xcolor}
\usepackage{listings}


\definecolor{codegreen}{rgb}{0,0.6,0}
\definecolor{codegray}{rgb}{0.5,0.5,0.5}
\definecolor{codepurple}{rgb}{0.58,0,0.82}
\definecolor{backcolour}{rgb}{0.95,0.95,0.92}

\lstdefinestyle{mystyle}{
    backgroundcolor=\color{backcolour},   
    commentstyle=\color{codegreen},
    keywordstyle=\color{magenta},
    numberstyle=\tiny\color{codegray},
    stringstyle=\color{codepurple},
    basicstyle=\ttfamily\footnotesize,
    breakatwhitespace=false,         
    breaklines=true,                 
    captionpos=b,                    
    keepspaces=true,                 
    numbers=left,                    
    numbersep=5pt,                  
    showspaces=false,                
    showstringspaces=false,
    showtabs=false,                  
    tabsize=2
}

\lstset{style=mystyle}

% Posiciona figuras e tabelas no topo da página quando adicionadas sozinhas em um página em branco
\makeatletter
\setlength{\@fptop}{5pt} 
\makeatother

% O tamanho do parágrafo é dado por
\setlength{\parindent}{1.3cm}

% Controle do espaçamento entre um parágrafo e outro
\setlength{\parskip}{0.2cm} 